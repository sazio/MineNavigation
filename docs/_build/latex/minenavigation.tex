%% Generated by Sphinx.
\def\sphinxdocclass{report}
\documentclass[letterpaper,10pt,english]{sphinxmanual}
\ifdefined\pdfpxdimen
   \let\sphinxpxdimen\pdfpxdimen\else\newdimen\sphinxpxdimen
\fi \sphinxpxdimen=.75bp\relax

\PassOptionsToPackage{warn}{textcomp}
\usepackage[utf8]{inputenc}
\ifdefined\DeclareUnicodeCharacter
% support both utf8 and utf8x syntaxes
  \ifdefined\DeclareUnicodeCharacterAsOptional
    \def\sphinxDUC#1{\DeclareUnicodeCharacter{"#1}}
  \else
    \let\sphinxDUC\DeclareUnicodeCharacter
  \fi
  \sphinxDUC{00A0}{\nobreakspace}
  \sphinxDUC{2500}{\sphinxunichar{2500}}
  \sphinxDUC{2502}{\sphinxunichar{2502}}
  \sphinxDUC{2514}{\sphinxunichar{2514}}
  \sphinxDUC{251C}{\sphinxunichar{251C}}
  \sphinxDUC{2572}{\textbackslash}
\fi
\usepackage{cmap}
\usepackage[T1]{fontenc}
\usepackage{amsmath,amssymb,amstext}
\usepackage{babel}



\usepackage{times}
\expandafter\ifx\csname T@LGR\endcsname\relax
\else
% LGR was declared as font encoding
  \substitutefont{LGR}{\rmdefault}{cmr}
  \substitutefont{LGR}{\sfdefault}{cmss}
  \substitutefont{LGR}{\ttdefault}{cmtt}
\fi
\expandafter\ifx\csname T@X2\endcsname\relax
  \expandafter\ifx\csname T@T2A\endcsname\relax
  \else
  % T2A was declared as font encoding
    \substitutefont{T2A}{\rmdefault}{cmr}
    \substitutefont{T2A}{\sfdefault}{cmss}
    \substitutefont{T2A}{\ttdefault}{cmtt}
  \fi
\else
% X2 was declared as font encoding
  \substitutefont{X2}{\rmdefault}{cmr}
  \substitutefont{X2}{\sfdefault}{cmss}
  \substitutefont{X2}{\ttdefault}{cmtt}
\fi


\usepackage[Bjarne]{fncychap}
\usepackage{sphinx}

\fvset{fontsize=\small}
\usepackage{geometry}

% Include hyperref last.
\usepackage{hyperref}
% Fix anchor placement for figures with captions.
\usepackage{hypcap}% it must be loaded after hyperref.
% Set up styles of URL: it should be placed after hyperref.
\urlstyle{same}
\addto\captionsenglish{\renewcommand{\contentsname}{Contents:}}

\usepackage{sphinxmessages}
\setcounter{tocdepth}{1}



\title{MineNavigation}
\date{Jun 19, 2019}
\release{1.0}
\author{Simone Azeglio}
\newcommand{\sphinxlogo}{\vbox{}}
\renewcommand{\releasename}{Release}
\makeindex
\begin{document}

\pagestyle{empty}
\sphinxmaketitle
\pagestyle{plain}
\sphinxtableofcontents
\pagestyle{normal}
\phantomsection\label{\detokenize{index::doc}}

\index{maze (module)@\spxentry{maze}\spxextra{module}}

\chapter{maze.py}
\label{\detokenize{index:maze-py}}

\bigskip\hrule\bigskip


\sphinxstylestrong{Main file} - it runs the mission

Basically, after loading the world (\sphinxstyleemphasis{.xml} file) it starts with two concatenated while statements (formally the mission is a concatenated while statement).
The second one represents each run. In the end it calculates the score and it saves a \sphinxstyleemphasis{.csv} file (log file) by setting the selected algorithm’s score.
We used \sphinxstyleemphasis{time.sleep(2)} in order to avoid issues related to items counting (stochastics fluctuations in time are significant and most of the times the set\_score
function wouldn’t work because of that, compromising the learning process)
\phantomsection\label{\detokenize{index:module-cli}}\index{cli (module)@\spxentry{cli}\spxextra{module}}

\chapter{cli.py}
\label{\detokenize{index:cli-py}}

\bigskip\hrule\bigskip


A file used for everything related to world, algorithms and files selection from terminal:
\index{valid\_algorithms() (in module cli)@\spxentry{valid\_algorithms()}\spxextra{in module cli}}

\begin{fulllineitems}
\phantomsection\label{\detokenize{index:cli.valid_algorithms}}\pysiglinewithargsret{\sphinxcode{\sphinxupquote{cli.}}\sphinxbfcode{\sphinxupquote{valid\_algorithms}}}{}{}
Shows valid algorithms (\sphinxstyleemphasis{genetic} and \sphinxstyleemphasis{hillclimbing} so far) and description

\end{fulllineitems}

\index{get\_algorithm() (in module cli)@\spxentry{get\_algorithm()}\spxextra{in module cli}}

\begin{fulllineitems}
\phantomsection\label{\detokenize{index:cli.get_algorithm}}\pysiglinewithargsret{\sphinxcode{\sphinxupquote{cli.}}\sphinxbfcode{\sphinxupquote{get\_algorithm}}}{\emph{alg\_selected}}{}
Returns a strategy based on the selected algorithm
(alg\_selected has to be one of the valid algorithms)

\end{fulllineitems}

\index{build\_maze\_filepath() (in module cli)@\spxentry{build\_maze\_filepath()}\spxextra{in module cli}}

\begin{fulllineitems}
\phantomsection\label{\detokenize{index:cli.build_maze_filepath}}\pysiglinewithargsret{\sphinxcode{\sphinxupquote{cli.}}\sphinxbfcode{\sphinxupquote{build\_maze\_filepath}}}{\emph{maze}}{}
It gets the filepath of the choosen \sphinxstyleemphasis{.xml} file

\end{fulllineitems}

\index{algorithms\_list() (in module cli)@\spxentry{algorithms\_list()}\spxextra{in module cli}}

\begin{fulllineitems}
\phantomsection\label{\detokenize{index:cli.algorithms_list}}\pysiglinewithargsret{\sphinxcode{\sphinxupquote{cli.}}\sphinxbfcode{\sphinxupquote{algorithms\_list}}}{}{}
Create list of algorithms for \sphinxstyleemphasis{parse\_args()} method

\end{fulllineitems}

\index{parse\_args() (in module cli)@\spxentry{parse\_args()}\spxextra{in module cli}}

\begin{fulllineitems}
\phantomsection\label{\detokenize{index:cli.parse_args}}\pysiglinewithargsret{\sphinxcode{\sphinxupquote{cli.}}\sphinxbfcode{\sphinxupquote{parse\_args}}}{}{}
Returns a triad : world (maze), algorithm and output file.
In this way it specifies the defined configuration.
The output file is used in order to let the agent build a memory and
to plot the fitness function

\end{fulllineitems}

\phantomsection\label{\detokenize{index:module-heuristics}}\index{heuristics (module)@\spxentry{heuristics}\spxextra{module}}

\chapter{heuristics.py}
\label{\detokenize{index:heuristics-py}}

\bigskip\hrule\bigskip


A file used for useful function:
\index{distance() (in module heuristics)@\spxentry{distance()}\spxextra{in module heuristics}}

\begin{fulllineitems}
\phantomsection\label{\detokenize{index:heuristics.distance}}\pysiglinewithargsret{\sphinxcode{\sphinxupquote{heuristics.}}\sphinxbfcode{\sphinxupquote{distance}}}{\emph{lhs}, \emph{rhs}}{}
Calculate the distance between left-hand-side (\sphinxstyleemphasis{lhs}) and right-hand-side (\sphinxstyleemphasis{rhs}).
E.g : lhs and rhs could be the agent and the diamond

\end{fulllineitems}

\index{location() (in module heuristics)@\spxentry{location()}\spxextra{in module heuristics}}

\begin{fulllineitems}
\phantomsection\label{\detokenize{index:heuristics.location}}\pysiglinewithargsret{\sphinxcode{\sphinxupquote{heuristics.}}\sphinxbfcode{\sphinxupquote{location}}}{\emph{entity}}{}
Define location as \sphinxstyleemphasis{x,y,z} triad

\end{fulllineitems}

\index{dot() (in module heuristics)@\spxentry{dot()}\spxextra{in module heuristics}}

\begin{fulllineitems}
\phantomsection\label{\detokenize{index:heuristics.dot}}\pysiglinewithargsret{\sphinxcode{\sphinxupquote{heuristics.}}\sphinxbfcode{\sphinxupquote{dot}}}{\emph{lhs}, \emph{rhs}}{}
Dot product between left-hand-side (\sphinxstyleemphasis{lhs}) and right-hand-side (\sphinxstyleemphasis{rhs})

\end{fulllineitems}

\index{diff() (in module heuristics)@\spxentry{diff()}\spxextra{in module heuristics}}

\begin{fulllineitems}
\phantomsection\label{\detokenize{index:heuristics.diff}}\pysiglinewithargsret{\sphinxcode{\sphinxupquote{heuristics.}}\sphinxbfcode{\sphinxupquote{diff}}}{\emph{lhs}, \emph{rhs}}{}
Coordinate difference between left-hand-side (\sphinxstyleemphasis{lhs}) and right-hand-side (\sphinxstyleemphasis{rhs})

\end{fulllineitems}

\index{magnitude() (in module heuristics)@\spxentry{magnitude()}\spxextra{in module heuristics}}

\begin{fulllineitems}
\phantomsection\label{\detokenize{index:heuristics.magnitude}}\pysiglinewithargsret{\sphinxcode{\sphinxupquote{heuristics.}}\sphinxbfcode{\sphinxupquote{magnitude}}}{\emph{vec}}{}
Magnitude of a vector (\sphinxstyleemphasis{vec})

\end{fulllineitems}

\index{normalize() (in module heuristics)@\spxentry{normalize()}\spxextra{in module heuristics}}

\begin{fulllineitems}
\phantomsection\label{\detokenize{index:heuristics.normalize}}\pysiglinewithargsret{\sphinxcode{\sphinxupquote{heuristics.}}\sphinxbfcode{\sphinxupquote{normalize}}}{\emph{vec}}{}
Normalize a vector (\sphinxstyleemphasis{vec})

\end{fulllineitems}

\index{get\_player\_location() (in module heuristics)@\spxentry{get\_player\_location()}\spxextra{in module heuristics}}

\begin{fulllineitems}
\phantomsection\label{\detokenize{index:heuristics.get_player_location}}\pysiglinewithargsret{\sphinxcode{\sphinxupquote{heuristics.}}\sphinxbfcode{\sphinxupquote{get\_player\_location}}}{\emph{el}}{}
Returns Agent’s location

\end{fulllineitems}

\index{get\_closest\_entity() (in module heuristics)@\spxentry{get\_closest\_entity()}\spxextra{in module heuristics}}

\begin{fulllineitems}
\phantomsection\label{\detokenize{index:heuristics.get_closest_entity}}\pysiglinewithargsret{\sphinxcode{\sphinxupquote{heuristics.}}\sphinxbfcode{\sphinxupquote{get\_closest\_entity}}}{\emph{el}, \emph{entity\_name}}{}
Returns the closest entity and the coordinate difference between agent and closest entity

\end{fulllineitems}

\index{closest\_cardinals() (in module heuristics)@\spxentry{closest\_cardinals()}\spxextra{in module heuristics}}

\begin{fulllineitems}
\phantomsection\label{\detokenize{index:heuristics.closest_cardinals}}\pysiglinewithargsret{\sphinxcode{\sphinxupquote{heuristics.}}\sphinxbfcode{\sphinxupquote{closest\_cardinals}}}{\emph{dir}, \emph{obs}}{}
Returns closest cardinal, in order to plan the path (dir = direction, obs = observation)

\end{fulllineitems}

\index{opposite\_direction() (in module heuristics)@\spxentry{opposite\_direction()}\spxextra{in module heuristics}}

\begin{fulllineitems}
\phantomsection\label{\detokenize{index:heuristics.opposite_direction}}\pysiglinewithargsret{\sphinxcode{\sphinxupquote{heuristics.}}\sphinxbfcode{\sphinxupquote{opposite\_direction}}}{\emph{dir}}{}
Returns opposite direction given a direction (dir = direction)

\end{fulllineitems}

\index{random\_direction() (in module heuristics)@\spxentry{random\_direction()}\spxextra{in module heuristics}}

\begin{fulllineitems}
\phantomsection\label{\detokenize{index:heuristics.random_direction}}\pysiglinewithargsret{\sphinxcode{\sphinxupquote{heuristics.}}\sphinxbfcode{\sphinxupquote{random\_direction}}}{\emph{obs}}{}
Returns a random direction in order to implement a random strategy (useful in order to don’t get stuck)
(obs = observation)

\end{fulllineitems}

\index{towards\_item() (in module heuristics)@\spxentry{towards\_item()}\spxextra{in module heuristics}}

\begin{fulllineitems}
\phantomsection\label{\detokenize{index:heuristics.towards_item}}\pysiglinewithargsret{\sphinxcode{\sphinxupquote{heuristics.}}\sphinxbfcode{\sphinxupquote{towards\_item}}}{\emph{obs}}{}
Returns the direction to follow in order to get to the item
(obs = observation)

\end{fulllineitems}

\phantomsection\label{\detokenize{index:module-mission}}\index{mission (module)@\spxentry{mission}\spxextra{module}}

\chapter{mission.py}
\label{\detokenize{index:mission-py}}

\bigskip\hrule\bigskip

\index{mission (class in mission)@\spxentry{mission}\spxextra{class in mission}}

\begin{fulllineitems}
\phantomsection\label{\detokenize{index:mission.mission}}\pysigline{\sphinxbfcode{\sphinxupquote{class }}\sphinxcode{\sphinxupquote{mission.}}\sphinxbfcode{\sphinxupquote{mission}}}
A class used for everything related to a mission (a run of the environment):
\index{load() (mission.mission method)@\spxentry{load()}\spxextra{mission.mission method}}

\begin{fulllineitems}
\phantomsection\label{\detokenize{index:mission.mission.load}}\pysiglinewithargsret{\sphinxbfcode{\sphinxupquote{load}}}{\emph{mission\_file}}{}
Load the world from \sphinxstyleemphasis{.xml} file and create default \sphinxstylestrong{Malmo} objects

\end{fulllineitems}

\index{start() (mission.mission method)@\spxentry{start()}\spxextra{mission.mission method}}

\begin{fulllineitems}
\phantomsection\label{\detokenize{index:mission.mission.start}}\pysiglinewithargsret{\sphinxbfcode{\sphinxupquote{start}}}{}{}
Set client pool, client info , reset the world for each new mission, set dimensions
of the video-window and agent’s viewpoint. Attempt to start a mission for 10 times if there’s any
issue occurring (some stochastics errors - take a look at Malmo’s official documentation:
\sphinxurl{https://github.com/microsoft/malmo} ), when the mission starts it keeps counting the time.

\end{fulllineitems}

\index{is\_running() (mission.mission method)@\spxentry{is\_running()}\spxextra{mission.mission method}}

\begin{fulllineitems}
\phantomsection\label{\detokenize{index:mission.mission.is_running}}\pysiglinewithargsret{\sphinxbfcode{\sphinxupquote{is\_running}}}{}{}
Check if the mission is currently running:
returns True if mission is running

\end{fulllineitems}

\index{get\_observation() (mission.mission method)@\spxentry{get\_observation()}\spxextra{mission.mission method}}

\begin{fulllineitems}
\phantomsection\label{\detokenize{index:mission.mission.get_observation}}\pysiglinewithargsret{\sphinxbfcode{\sphinxupquote{get\_observation}}}{}{}
Loads floor grid, edge distances, current player position, and entity (e.g. diamond) position
in order to let the agent know the distance to the diamond (odor-like representation).
It also counts collected items and exports the world view in a \sphinxstyleemphasis{.json} file.
Agent’s observation is a \sphinxstyleemphasis{3x3 grid}: he’s in the center and he knows only 1 block in every
possible direction

\end{fulllineitems}

\index{send\_command() (mission.mission method)@\spxentry{send\_command()}\spxextra{mission.mission method}}

\begin{fulllineitems}
\phantomsection\label{\detokenize{index:mission.mission.send_command}}\pysiglinewithargsret{\sphinxbfcode{\sphinxupquote{send\_command}}}{}{}
Send command input to the Agent in order to perform the next move.
It even counts if the current cell is a newly explored one or not (this
will be useful in implementing the score function)

\end{fulllineitems}

\index{stop\_clock() (mission.mission method)@\spxentry{stop\_clock()}\spxextra{mission.mission method}}

\begin{fulllineitems}
\phantomsection\label{\detokenize{index:mission.mission.stop_clock}}\pysiglinewithargsret{\sphinxbfcode{\sphinxupquote{stop\_clock}}}{}{}
It stops the time at the end of the mission (useful for score function based on time)

\end{fulllineitems}

\index{check\_errors() (mission.mission method)@\spxentry{check\_errors()}\spxextra{mission.mission method}}

\begin{fulllineitems}
\phantomsection\label{\detokenize{index:mission.mission.check_errors}}\pysiglinewithargsret{\sphinxbfcode{\sphinxupquote{check\_errors}}}{}{}
Check whether there are errors in the mission (check \sphinxstyleemphasis{world\_state} Malmo’s method). Those errors are typically related to Malmo Client

\end{fulllineitems}

\index{block\_score() (mission.mission method)@\spxentry{block\_score()}\spxextra{mission.mission method}}

\begin{fulllineitems}
\phantomsection\label{\detokenize{index:mission.mission.block_score}}\pysiglinewithargsret{\sphinxbfcode{\sphinxupquote{block\_score}}}{}{}
The agent has to explore new blocks in order to get a better \sphinxstyleemphasis{fitness}, he gets a penalty by being in the same
block (no points for fitness)

\end{fulllineitems}

\index{time\_score() (mission.mission method)@\spxentry{time\_score()}\spxextra{mission.mission method}}

\begin{fulllineitems}
\phantomsection\label{\detokenize{index:mission.mission.time_score}}\pysiglinewithargsret{\sphinxbfcode{\sphinxupquote{time\_score}}}{}{}
The agent gets one fitness point for each seconds he spends alive

\end{fulllineitems}

\index{item\_score() (mission.mission method)@\spxentry{item\_score()}\spxextra{mission.mission method}}

\begin{fulllineitems}
\phantomsection\label{\detokenize{index:mission.mission.item_score}}\pysiglinewithargsret{\sphinxbfcode{\sphinxupquote{item\_score}}}{}{}
The agent gets some more points for each item he picks up from the ground
(1 diamond = 50 fitness points)

\end{fulllineitems}

\index{score() (mission.mission method)@\spxentry{score()}\spxextra{mission.mission method}}

\begin{fulllineitems}
\phantomsection\label{\detokenize{index:mission.mission.score}}\pysiglinewithargsret{\sphinxbfcode{\sphinxupquote{score}}}{}{}
Sums up block\_score , time\_score and item\_score to get the Fitness

\end{fulllineitems}


\end{fulllineitems}

\index{observation (class in mission)@\spxentry{observation}\spxextra{class in mission}}

\begin{fulllineitems}
\phantomsection\label{\detokenize{index:mission.observation}}\pysiglinewithargsret{\sphinxbfcode{\sphinxupquote{class }}\sphinxcode{\sphinxupquote{mission.}}\sphinxbfcode{\sphinxupquote{observation}}}{\emph{set\_grid}, \emph{set\_edge\_distances}, \emph{set\_cell}, \emph{set\_entity\_locations}}{}
A class used to represent an observation of the world:
\index{set\_grid (mission.observation attribute)@\spxentry{set\_grid}\spxextra{mission.observation attribute}}

\begin{fulllineitems}
\phantomsection\label{\detokenize{index:mission.observation.set_grid}}\pysigline{\sphinxbfcode{\sphinxupquote{set\_grid}}}
Extracted from the \sphinxstyleemphasis{.json} World File by \sphinxstyleemphasis{mission.get\_observation()}
The grid is the part of the world seen by the agent
\begin{quote}\begin{description}
\item[{Type}] \leavevmode
str

\end{description}\end{quote}

\end{fulllineitems}

\index{set\_edge\_distances (mission.observation attribute)@\spxentry{set\_edge\_distances}\spxextra{mission.observation attribute}}

\begin{fulllineitems}
\phantomsection\label{\detokenize{index:mission.observation.set_edge_distances}}\pysigline{\sphinxbfcode{\sphinxupquote{set\_edge\_distances}}}
Extracted from the \sphinxstyleemphasis{.json} World File by \sphinxstyleemphasis{mission.get\_observation()}
Distances from the edges of the world (The world is limited)
\begin{quote}\begin{description}
\item[{Type}] \leavevmode
str

\end{description}\end{quote}

\end{fulllineitems}

\index{set\_cell (mission.observation attribute)@\spxentry{set\_cell}\spxextra{mission.observation attribute}}

\begin{fulllineitems}
\phantomsection\label{\detokenize{index:mission.observation.set_cell}}\pysigline{\sphinxbfcode{\sphinxupquote{set\_cell}}}
Extracted from the \sphinxstyleemphasis{.json} World File by \sphinxstyleemphasis{mission.get\_observation()}
Cell is the agent location in the world (The agent has a GPS)
\begin{quote}\begin{description}
\item[{Type}] \leavevmode
str

\end{description}\end{quote}

\end{fulllineitems}

\index{set\_entity\_locations (mission.observation attribute)@\spxentry{set\_entity\_locations}\spxextra{mission.observation attribute}}

\begin{fulllineitems}
\phantomsection\label{\detokenize{index:mission.observation.set_entity_locations}}\pysigline{\sphinxbfcode{\sphinxupquote{set\_entity\_locations}}}
Extracted from the .json World File by mission.get\_observation()
Locations of entities (e.g. Diamonds, Zombies)
\begin{quote}\begin{description}
\item[{Type}] \leavevmode
str

\end{description}\end{quote}

\end{fulllineitems}

\index{at\_junction() (mission.observation method)@\spxentry{at\_junction()}\spxextra{mission.observation method}}

\begin{fulllineitems}
\phantomsection\label{\detokenize{index:mission.observation.at_junction}}\pysiglinewithargsret{\sphinxbfcode{\sphinxupquote{at\_junction}}}{}{}
States whether a move is plausible or not : The agent can only walk over glowstone.
This is useful in order to limit the world with another material (e.g. Lava)

\end{fulllineitems}


\end{fulllineitems}

\phantomsection\label{\detokenize{index:module-algorithms.algorithm}}\index{algorithms.algorithm (module)@\spxentry{algorithms.algorithm}\spxextra{module}}

\chapter{algorithm.py}
\label{\detokenize{index:algorithm-py}}

\bigskip\hrule\bigskip

\index{algorithm (class in algorithms.algorithm)@\spxentry{algorithm}\spxextra{class in algorithms.algorithm}}

\begin{fulllineitems}
\phantomsection\label{\detokenize{index:algorithms.algorithm.algorithm}}\pysiglinewithargsret{\sphinxbfcode{\sphinxupquote{class }}\sphinxcode{\sphinxupquote{algorithms.algorithm.}}\sphinxbfcode{\sphinxupquote{algorithm}}}{\emph{set\_actions}}{}
A class used to define an algorithm and its actions:
\begin{description}
\item[{Attributes}] \leavevmode
\end{description}
\index{process\_score() (algorithms.algorithm.algorithm method)@\spxentry{process\_score()}\spxextra{algorithms.algorithm.algorithm method}}

\begin{fulllineitems}
\phantomsection\label{\detokenize{index:algorithms.algorithm.algorithm.process_score}}\pysiglinewithargsret{\sphinxbfcode{\sphinxupquote{process\_score}}}{\emph{score}}{}
\sphinxstyleemphasis{@abc.abstractmethod}
After specifying the algorithm, while running the program, it processes
the score by following the current algorithm rule (\sphinxstyleemphasis{genetic, hillclimb})

\end{fulllineitems}

\index{set\_score() (algorithms.algorithm.algorithm method)@\spxentry{set\_score()}\spxextra{algorithms.algorithm.algorithm method}}

\begin{fulllineitems}
\phantomsection\label{\detokenize{index:algorithms.algorithm.algorithm.set_score}}\pysiglinewithargsret{\sphinxbfcode{\sphinxupquote{set\_score}}}{\emph{score}}{}
When each run of a mission is ended it sets the score and it saves the score in a
\sphinxstyleemphasis{.csv} file (useful for plotting the fitness function)

\end{fulllineitems}

\index{get\_action() (algorithms.algorithm.algorithm method)@\spxentry{get\_action()}\spxextra{algorithms.algorithm.algorithm method}}

\begin{fulllineitems}
\phantomsection\label{\detokenize{index:algorithms.algorithm.algorithm.get_action}}\pysiglinewithargsret{\sphinxbfcode{\sphinxupquote{get\_action}}}{\emph{obs}}{}
\sphinxstyleemphasis{@abc.abstractmethod}
It gets the action that the agent has to perform from the specific algorithm (\sphinxstyleemphasis{genetic, hillclimb})

\end{fulllineitems}


\end{fulllineitems}

\phantomsection\label{\detokenize{index:module-algorithms.genetic}}\index{algorithms.genetic (module)@\spxentry{algorithms.genetic}\spxextra{module}}

\chapter{genetic.py}
\label{\detokenize{index:genetic-py}}

\bigskip\hrule\bigskip


\sphinxstylestrong{Pseudocode for the creation of a new population:}

\sphinxstyleemphasis{fittest = four top scoring strings within the population}

\begin{DUlineblock}{0em}
\item[] \sphinxstyleemphasis{for i in range(population\_size):}
\end{DUlineblock}
\begin{quote}

\begin{DUlineblock}{0em}
\item[] \sphinxstyleemphasis{parent1 = random.choice(fittest)}
\end{DUlineblock}

\begin{DUlineblock}{0em}
\item[] \sphinxstyleemphasis{parent2 = random.choice(fittest)}
\end{DUlineblock}

\begin{DUlineblock}{0em}
\item[] \sphinxstyleemphasis{offspring = parent1{[}crossover:{]} + parent2{[}:crossover{]}}
\end{DUlineblock}

\begin{DUlineblock}{0em}
\item[] \sphinxstyleemphasis{for heuristic in offspring:}
\end{DUlineblock}
\begin{quote}

\begin{DUlineblock}{0em}
\item[] \sphinxstyleemphasis{5\% chance to mutate heuristic to another one}
\end{DUlineblock}
\end{quote}

\begin{DUlineblock}{0em}
\item[] \sphinxstyleemphasis{population.append(offspring)}
\end{DUlineblock}
\end{quote}
\index{genetic (class in algorithms.genetic)@\spxentry{genetic}\spxextra{class in algorithms.genetic}}

\begin{fulllineitems}
\phantomsection\label{\detokenize{index:algorithms.genetic.genetic}}\pysiglinewithargsret{\sphinxbfcode{\sphinxupquote{class }}\sphinxcode{\sphinxupquote{algorithms.genetic.}}\sphinxbfcode{\sphinxupquote{genetic}}}{\emph{set\_actions}, \emph{set\_gen\_size=8}, \emph{set\_str\_len=5}, \emph{set\_sel\_frac=0.5}, \emph{set\_mut\_prob=0.05}}{}
A class used to define the genetic algorithm:
\index{set\_actions (algorithms.genetic.genetic attribute)@\spxentry{set\_actions}\spxextra{algorithms.genetic.genetic attribute}}

\begin{fulllineitems}
\phantomsection\label{\detokenize{index:algorithms.genetic.genetic.set_actions}}\pysiglinewithargsret{\sphinxbfcode{\sphinxupquote{set\_actions}}}{\emph{str}}{}
list of possible actions the agent can take

\end{fulllineitems}

\index{set\_gen\_size (algorithms.genetic.genetic attribute)@\spxentry{set\_gen\_size}\spxextra{algorithms.genetic.genetic attribute}}

\begin{fulllineitems}
\phantomsection\label{\detokenize{index:algorithms.genetic.genetic.set_gen_size}}\pysiglinewithargsret{\sphinxbfcode{\sphinxupquote{set\_gen\_size}}}{\emph{int}}{}
Size of the generation, each generation is a list of strategies (t = towards the entity, the diamond ; r = random move).
Each generation has 8 list strategies in this model

\end{fulllineitems}

\index{set\_str\_len (algorithms.genetic.genetic attribute)@\spxentry{set\_str\_len}\spxextra{algorithms.genetic.genetic attribute}}

\begin{fulllineitems}
\phantomsection\label{\detokenize{index:algorithms.genetic.genetic.set_str_len}}\pysiglinewithargsret{\sphinxbfcode{\sphinxupquote{set\_str\_len}}}{\emph{int}}{}
Length of the list, it starts with an average (gaussian distribution) of length 5

\end{fulllineitems}

\index{set\_sel\_frac (algorithms.genetic.genetic attribute)@\spxentry{set\_sel\_frac}\spxextra{algorithms.genetic.genetic attribute}}

\begin{fulllineitems}
\phantomsection\label{\detokenize{index:algorithms.genetic.genetic.set_sel_frac}}\pysiglinewithargsret{\sphinxbfcode{\sphinxupquote{set\_sel\_frac}}}{\emph{double}}{}
Sets the selected fraction of the most 4 top high-scoring strings in order to generate strings in the next iteration

\end{fulllineitems}

\index{set\_mut\_prob (algorithms.genetic.genetic attribute)@\spxentry{set\_mut\_prob}\spxextra{algorithms.genetic.genetic attribute}}

\begin{fulllineitems}
\phantomsection\label{\detokenize{index:algorithms.genetic.genetic.set_mut_prob}}\pysiglinewithargsret{\sphinxbfcode{\sphinxupquote{set\_mut\_prob}}}{\emph{double}}{}
Mutation probability of the genetic algorithm (p = 0.05)

\end{fulllineitems}

\index{next\_generation() (algorithms.genetic.genetic method)@\spxentry{next\_generation()}\spxextra{algorithms.genetic.genetic method}}

\begin{fulllineitems}
\phantomsection\label{\detokenize{index:algorithms.genetic.genetic.next_generation}}\pysiglinewithargsret{\sphinxbfcode{\sphinxupquote{next\_generation}}}{}{}
Creates next generation of strategies. It finds the best scores in the population, it selects the top 4 high-scoring strings
in order to generate the next generation of strings.

\end{fulllineitems}

\index{process\_score() (algorithms.genetic.genetic method)@\spxentry{process\_score()}\spxextra{algorithms.genetic.genetic method}}

\begin{fulllineitems}
\phantomsection\label{\detokenize{index:algorithms.genetic.genetic.process_score}}\pysiglinewithargsret{\sphinxbfcode{\sphinxupquote{process\_score}}}{\emph{score}}{}
Keeps track of how the score changes and improves at each iteration

\end{fulllineitems}

\index{get\_action() (algorithms.genetic.genetic method)@\spxentry{get\_action()}\spxextra{algorithms.genetic.genetic method}}

\begin{fulllineitems}
\phantomsection\label{\detokenize{index:algorithms.genetic.genetic.get_action}}\pysiglinewithargsret{\sphinxbfcode{\sphinxupquote{get\_action}}}{\emph{obs}}{}
Returns an action based on observations and on a policy which considers the previous methods

\end{fulllineitems}


\end{fulllineitems}

\phantomsection\label{\detokenize{index:module-algorithms.hillclimbing}}\index{algorithms.hillclimbing (module)@\spxentry{algorithms.hillclimbing}\spxextra{module}}

\chapter{hillclimbing.py}
\label{\detokenize{index:hillclimbing-py}}

\bigskip\hrule\bigskip


After scoring the first string, the algorithm runs a mission for each string adjacent to the first string in the search space.
An adjacent string is a string which differs by only one addition of a heuristic from the string, removal of a heuristic, or change of a heuristic.
After scoring every adjacent string, the algorithm chooses the string with the best score. It then explores the adjacent strings to that string,
choosing the best one of those, and so on. These incremental improvements allow the algorithm to find heuristic strings that produce higher and higher scores.

Added \sphinxstyleemphasis{Simulated Annealing} feature!
The purpose of this probabilistic behavior is to maximize the space that the hill-climbing algorithm explores.
Rather than sticking with whatever seems locally optimal,  the hill-climbing algorithm may find even better strings
in areas of the search space that, at first glance, seemed sub-optimal.

\sphinxstylestrong{Pseudocode for hillclimbing:}
\begin{quote}
\begin{description}
\item[{\sphinxstyleemphasis{while True:}}] \leavevmode\begin{description}
\item[{\sphinxstyleemphasis{for string adjacent to current\_string:}}] \leavevmode\begin{description}
\item[{\sphinxstyleemphasis{if score(string) \textgreater{} score(current\_string):}}] \leavevmode
\sphinxstyleemphasis{best\_string = string}

\end{description}

\end{description}

\sphinxstyleemphasis{current\_string = best\_string}

\end{description}
\end{quote}

\sphinxstylestrong{Pseudocode for Simulated Annealing:}
\begin{quote}
\begin{description}
\item[{\sphinxstyleemphasis{while True:}}] \leavevmode
\sphinxstyleemphasis{prob = probability that we choose a suboptimal choice}

\sphinxstyleemphasis{eps = random.random()}

\sphinxstyleemphasis{cooling\_rate = 0.5}

\sphinxstyleemphasis{neighbors = every string adjacent to current\_string}
\begin{description}
\item[{\sphinxstyleemphasis{if eps \textless{} p:}}] \leavevmode
\sphinxstyleemphasis{string = random.choice(neighbors)}

\item[{\sphinxstyleemphasis{else:}}] \leavevmode\begin{description}
\item[{\sphinxstyleemphasis{for string adjacent to current\_string:}}] \leavevmode\begin{description}
\item[{\sphinxstyleemphasis{if score(string) \textgreater{} score(current\_string):}}] \leavevmode
\sphinxstyleemphasis{best\_string = string}

\end{description}

\end{description}

\sphinxstyleemphasis{current\_string = best\_string}

\sphinxstyleemphasis{prob *= cooling\_rate}

\end{description}

\end{description}
\end{quote}
\index{climber (class in algorithms.hillclimbing)@\spxentry{climber}\spxextra{class in algorithms.hillclimbing}}

\begin{fulllineitems}
\phantomsection\label{\detokenize{index:algorithms.hillclimbing.climber}}\pysiglinewithargsret{\sphinxbfcode{\sphinxupquote{class }}\sphinxcode{\sphinxupquote{algorithms.hillclimbing.}}\sphinxbfcode{\sphinxupquote{climber}}}{\emph{set\_actions}, \emph{init\_eps=0.0}, \emph{set\_cooling=1.0}}{}
A class used to define the hillclimbing algorithm:
\index{set\_actions (algorithms.hillclimbing.climber attribute)@\spxentry{set\_actions}\spxextra{algorithms.hillclimbing.climber attribute}}

\begin{fulllineitems}
\phantomsection\label{\detokenize{index:algorithms.hillclimbing.climber.set_actions}}\pysiglinewithargsret{\sphinxbfcode{\sphinxupquote{set\_actions}}}{\emph{str}}{}
List of possible actions the agent can take

\end{fulllineitems}

\index{init\_eps (algorithms.hillclimbing.climber attribute)@\spxentry{init\_eps}\spxextra{algorithms.hillclimbing.climber attribute}}

\begin{fulllineitems}
\phantomsection\label{\detokenize{index:algorithms.hillclimbing.climber.init_eps}}\pysiglinewithargsret{\sphinxbfcode{\sphinxupquote{init\_eps}}}{\emph{double}}{}
Minimum reduction in the function before termination (Simulated Annealing - \sphinxurl{https://www.aero.iitb.ac.in/~rkpant/webpages/DefaultWebApp/salect.pdf})

\end{fulllineitems}

\index{set\_cooling (algorithms.hillclimbing.climber attribute)@\spxentry{set\_cooling}\spxextra{algorithms.hillclimbing.climber attribute}}

\begin{fulllineitems}
\phantomsection\label{\detokenize{index:algorithms.hillclimbing.climber.set_cooling}}\pysiglinewithargsret{\sphinxbfcode{\sphinxupquote{set\_cooling}}}{\emph{double}}{}
Cooling rate (Simulated Annealing - \sphinxurl{https://www.aero.iitb.ac.in/~rkpant/webpages/DefaultWebApp/salect.pdf})

\end{fulllineitems}

\index{generate\_local\_space() (algorithms.hillclimbing.climber method)@\spxentry{generate\_local\_space()}\spxextra{algorithms.hillclimbing.climber method}}

\begin{fulllineitems}
\phantomsection\label{\detokenize{index:algorithms.hillclimbing.climber.generate_local_space}}\pysiglinewithargsret{\sphinxbfcode{\sphinxupquote{generate\_local\_space}}}{}{}
Starting from a string it performs the three possible operations on heuristics creating a local search space.
After that it starts the Simulated Annealing algorithm

\end{fulllineitems}

\index{pick\_next\_string() (algorithms.hillclimbing.climber method)@\spxentry{pick\_next\_string()}\spxextra{algorithms.hillclimbing.climber method}}

\begin{fulllineitems}
\phantomsection\label{\detokenize{index:algorithms.hillclimbing.climber.pick_next_string}}\pysiglinewithargsret{\sphinxbfcode{\sphinxupquote{pick\_next\_string}}}{}{}
It selects the next string, starting from the best score string and looking around that one in the search space

\end{fulllineitems}

\index{process\_score() (algorithms.hillclimbing.climber method)@\spxentry{process\_score()}\spxextra{algorithms.hillclimbing.climber method}}

\begin{fulllineitems}
\phantomsection\label{\detokenize{index:algorithms.hillclimbing.climber.process_score}}\pysiglinewithargsret{\sphinxbfcode{\sphinxupquote{process\_score}}}{\emph{score}}{}
Keeps track of how the score changes for ending in the same block. It returns a combination of score and
current string heuristics

\end{fulllineitems}

\index{get\_action() (algorithms.hillclimbing.climber method)@\spxentry{get\_action()}\spxextra{algorithms.hillclimbing.climber method}}

\begin{fulllineitems}
\phantomsection\label{\detokenize{index:algorithms.hillclimbing.climber.get_action}}\pysiglinewithargsret{\sphinxbfcode{\sphinxupquote{get\_action}}}{\emph{obs}}{}
Returns action based on heuristics string (which is based on observations)

\end{fulllineitems}


\end{fulllineitems}



\chapter{Indices and tables}
\label{\detokenize{index:indices-and-tables}}\begin{itemize}
\item {} 
\DUrole{xref,std,std-ref}{genindex}

\item {} 
\DUrole{xref,std,std-ref}{modindex}

\item {} 
\DUrole{xref,std,std-ref}{search}

\end{itemize}


\renewcommand{\indexname}{Python Module Index}
\begin{sphinxtheindex}
\let\bigletter\sphinxstyleindexlettergroup
\bigletter{a}
\item\relax\sphinxstyleindexentry{algorithms.algorithm}\sphinxstyleindexpageref{index:\detokenize{module-algorithms.algorithm}}
\item\relax\sphinxstyleindexentry{algorithms.genetic}\sphinxstyleindexpageref{index:\detokenize{module-algorithms.genetic}}
\item\relax\sphinxstyleindexentry{algorithms.hillclimbing}\sphinxstyleindexpageref{index:\detokenize{module-algorithms.hillclimbing}}
\indexspace
\bigletter{c}
\item\relax\sphinxstyleindexentry{cli}\sphinxstyleindexpageref{index:\detokenize{module-cli}}
\indexspace
\bigletter{h}
\item\relax\sphinxstyleindexentry{heuristics}\sphinxstyleindexpageref{index:\detokenize{module-heuristics}}
\indexspace
\bigletter{m}
\item\relax\sphinxstyleindexentry{maze}\sphinxstyleindexpageref{index:\detokenize{module-maze}}
\item\relax\sphinxstyleindexentry{mission}\sphinxstyleindexpageref{index:\detokenize{module-mission}}
\end{sphinxtheindex}

\renewcommand{\indexname}{Index}
\printindex
\end{document}